\documentclass[a4paper]{article}

\setlength{\parskip}{2mm}
\setlength{\parindent}{0pt} % Make  the default
\newcommand{\tab}{~ \qquad}
\input{Macros}
\usepackage{caratula} % Version modificada para usar las macros de algo1 de ~> https://github.com/bcardiff/dc-tex
\usepackage{graphicx}
\usepackage{booktabs}
\usepackage{amsmath}
\usepackage{amssymb}

\begin{document}

\titulo{Resolviendo la Ecuación de Onda mediante la Transformada de Fourier Acelerada}
% \subtitulo{Estudiando la performance de }
\fecha{30 de Agosto de 2025}
\materia{Organización del Computador II}

\integrante{Polonuer, Joaquin}{1612/21}{jtpolonuer@gmail.com}

\maketitle

\tableofcontents
\newpage

\section{Introducción}

\subsection{La Ecuación de Onda}

La ecuación de onda representa uno de los fenómenos físicos más fundamentales en la naturaleza, describiendo la propagación de
perturbaciones en medios continuos. Desde ondas sonoras y electromagnéticas hasta vibraciones mecánicas, este modelo matemático
encuentra aplicación en campos tan diversos como la acústica, la óptica, la sismología y la ingeniería estructural.

La ecuación de onda unidimensional se expresa como:

\begin{equation}
    \frac{\partial^2 u}{\partial t^2} = c^2 \frac{\partial^2 u}{\partial x^2}
\end{equation}

donde $u(x,t)$ representa el desplazamiento de la onda en el punto $x$ y tiempo $t$, y $c$ es la velocidad de propagación característica
del medio. Esta ecuación en derivadas parciales de segundo orden describe fenómenos como:

\begin{itemize}
    \item Vibraciones de cuerdas tensadas
    \item Propagación de ondas sonoras en tubos
    \item Ondas electromagnéticas en líneas de transmisión
\end{itemize}

Naturalmente, su extensión a dos dimensiones espaciales es:

\begin{equation}
    \frac{\partial^2 u}{\partial t^2} = c^2 \left(\frac{\partial^2 u}{\partial x^2} + \frac{\partial^2 u}{\partial y^2}\right) = c^2 \nabla^2 u
\end{equation}

Esta formulación bidimensional modela fenómenos como:

\begin{itemize}
    \item Vibraciones de membranas (tambores, diafragmas)
    \item Ondas superficiales en líquidos
    \item Propagación de ondas sísmicas en planos
    \item Ondas electromagnéticas
\end{itemize}

\subsection{La Transformada de Fourier}

La Transformada de Fourier constituye una herramienta matemática fundamental para el análisis de fenómenos ondulatorios, permitiendo
descomponer señales complejas en sus componentes frecuenciales básicas. Esta transformación resulta especialmente poderosa en el
contexto de la resolución de ecuaciones diferenciales parciales.

La Transformada de Fourier continua de una función $f(x)$ se define como:

\begin{equation}
    \hat{f}(\omega) = \int_{-\infty}^{\infty} f(x) e^{-i\omega x} dx
\end{equation}

y su transformada inversa:

\begin{equation}
    f(x) = \frac{1}{2\pi} \int_{-\infty}^{\infty} \hat{f}(\omega) e^{i\omega x} d\omega
\end{equation}


\subsection{Resolución de la Ecuación de Onda mediante la Transformada de Fourier}

La aplicación de la Transformada de Fourier a la ecuación de onda permite convertir la ecuación en derivadas parciales en una ecuación diferencial ordinaria,
lo que simplifica notablemente su resolución. Para abordar la ecuación (1), se aprovechan propiedades fundamentales de la transformada de Fourier relacionadas con las derivadas:

% Props
\textbf{Propiedad 1:} La transformada de Fourier de la derivada $n$-ésima de una función es igual a la multiplicación por $(i\omega)^n$ de la transformada de la función original. En particular, para $n=1$ se obtiene el caso de la derivada simple.

\begin{equation}
    \mathcal{F}\left\{\frac{\partial^n f}{\partial x^n}\right\} = (i\omega)^n \hat{f}(\omega)
\end{equation}

\textbf{Propiedad 2:} Cuando se transforma respecto a una variable diferente a la que se deriva, la derivada parcial se convierte en una derivada ordinaria de la transformada:

\begin{equation}
    \mathcal{F}_x\left\{\frac{\partial f}{\partial t}\right\} = \frac{\partial \hat{f}}{\partial t}
\end{equation}

donde $\mathcal{F}_x$ denota la transformada de Fourier respecto a la variable $x$.
% fin props

Es por estas dos propiedades que al aplicar la Transformada de Fourier espacial, obtenemos:

\begin{equation}
    \frac{\partial^2 \hat{u}}{\partial t^2} = -c^2 \omega^2 \hat{u}
\end{equation}

donde $\hat{u}(\omega, t)$ es la transformada de Fourier de $u(x,t)$ respecto a $x$.

Que es ecuación diferencial ordinaria con solución analitica conocida:

\begin{equation}
    \hat{u}(\omega, t) = A(\omega) e^{ic\omega t} + B(\omega) e^{-ic\omega t}
\end{equation}

Los coeficientes $A(\omega)$ y $B(\omega)$ se determinan a partir de las condiciones iniciales, y la solución final se obtiene
aplicando la transformada inversa.

\subsection{Transformada Discreta de Fourier}

Para implementaciones computacionales, la Transformada de Fourier continua debe discretizarse. La Transformada Discreta de Fourier
(DFT) de una secuencia finita $x[n]$ de $N$ elementos se define como:

\begin{equation}
    \hat{x}[k] = \sum_{n=0}^{N-1} x[n] e^{-i2\pi kn/N}, \quad k = 0, 1, \ldots, N-1
\end{equation}

donde $\hat{x}[k]$ representa los coeficientes espectrales discretos. La transformada inversa se expresa como:

\begin{equation}
    x[n] = \frac{1}{N} \sum_{k=0}^{N-1} \hat{x}[k] e^{i2\pi kn/N}, \quad n = 0, 1, \ldots, N-1
\end{equation}

La implementación directa de la DFT requiere $O(N^2)$ operaciones complejas, lo que resulta computacionalmente prohibitivo para
secuencias largas. Esta limitación motivó el desarrollo del algoritmo Fast Fourier Transform.

\subsection{Transformada Rápida de Fourier (Cooley-Tukey)}

El algoritmo FFT, desarrollado por Cooley y Tukey en 1965, reduce la complejidad computacional de $O(N^2)$ a $O(N \log N)$
mediante la estrategia de divide and conquer. Para $N = 2^m$, el algoritmo descompone la DFT en DFTs más pequeñas.

El algoritmo DIT (Decimation-in-Time) separa la secuencia de entrada en muestras pares e impares:

\begin{equation}
    \hat{x}[k] = \sum_{n \text{ par}} x[n] e^{-i2\pi kn/N} + \sum_{n \text{ impar}} x[n] e^{-i2\pi kn/N}
\end{equation}

Sustituyendo $n = 2r$ para índices pares y $n = 2r+1$ para impares:

\begin{equation}
    \hat{x}[k] = \sum_{r=0}^{N/2-1} x[2r] e^{-i2\pi kr/(N/2)} + e^{-i2\pi k/N} \sum_{r=0}^{N/2-1} x[2r+1] e^{-i2\pi kr/(N/2)}
\end{equation}

Definiendo:
\begin{align}
    \hat{x}_{\text{par}}[k]   & = \sum_{r=0}^{N/2-1} x[2r] e^{-i2\pi kr/(N/2)}   \\
    \hat{x}_{\text{impar}}[k] & = \sum_{r=0}^{N/2-1} x[2r+1] e^{-i2\pi kr/(N/2)}
\end{align}

La ecuación se simplifica a:

\begin{equation}
    \hat{x}[k] = \hat{x}_{\text{par}}[k] + W_N^k \cdot \hat{x}_{\text{impar}}[k]
\end{equation}

donde $W_N^k = e^{-i2\pi k/N}$ es el factor de giro (twiddle factor).

Aprovechando la periodicidad $\hat{x}_{\text{par}}[k + N/2] = \hat{x}_{\text{par}}[k]$ y la simetría $W_N^{k+N/2} = -W_N^k$:

\begin{align}
    \hat{x}[k]       & = \hat{x}_{\text{par}}[k] + W_N^k \cdot \hat{x}_{\text{impar}}[k] \\
    \hat{x}[k + N/2] & = \hat{x}_{\text{par}}[k] - W_N^k \cdot \hat{x}_{\text{impar}}[k]
\end{align}

Este proceso se aplica recursivamente hasta obtener DFTs de un solo elemento.

\subsubsection{Algoritmo Recursivo FFT}

El algoritmo recursivo implementa directamente la estrategia divide and conquer descrita anteriormente:

\begin{verbatim}
def fft_recursive(x):
    n = len(x)
    if n <= 1:
        return x
    
    # Dividir en pares e impares
    x_par = [x[2*i] for i in range(n//2)]
    x_impar = [x[2*i+1] for i in range(n//2)]
    
    # Aplicar FFT recursivamente
    y_par = fft_recursive(x_par)
    y_impar = fft_recursive(x_impar)
    
    # Combinar resultados
    y = [0] * n
    for k in range(n//2):
        w = exp(-2j * pi * k / n)
        y[k] = y_par[k] + w * y_impar[k]
        y[k + n//2] = y_par[k] - w * y_impar[k]
    
    return y
\end{verbatim}

Este algoritmo recursivo tiene complejidad $O(N \log N)$ pero presenta overhead significativo debido a:
\begin{itemize}
    \item Creación de múltiples listas temporales
    \item Llamadas recursivas con overhead de stack
    \item Acceso no secuencial a memoria
\end{itemize}

Por esta razón, las implementaciones prácticas utilizan versiones iterativas optimizadas que mantienen la misma complejidad algorítmica pero con mejor rendimiento en la práctica.

\subsubsection{Algoritmo Iterativo FFT}

La versión iterativa evita el overhead de la recursión mediante el uso de bit-reversal y bucles anidados:

\begin{verbatim}
def fft_iterative(x):
    n = len(x)
    if n <= 1:
        return x
    
    # Bit-reversal permutation
    j = 0
    for i in range(1, n):
        bit = n >> 1
        while j & bit:
            j ^= bit
            bit >>= 1
        j ^= bit
        if i < j:
            x[i], x[j] = x[j], x[i]
    
    # FFT iterativa
    length = 2
    while length <= n:
        w = exp(-2j * pi / length)
        for i in range(0, n, length):
            wn = 1 + 0j
            for j in range(length // 2):
                u = x[i + j]
                v = x[i + j + length // 2] * wn
                x[i + j] = u + v
                x[i + j + length // 2] = u - v
                wn *= w
        length <<= 1
    
    return x
\end{verbatim}

Esta implementación iterativa presenta varias ventajas sobre la versión recursiva:

\begin{itemize}
    \item \textbf{Acceso secuencial a memoria}: Mejor utilización de la jerarquía de cache
    \item \textbf{Sin overhead de recursión}: Elimina el costo de las llamadas a función
    \item \textbf{Menor uso de memoria}: No requiere múltiples copias de los datos
    \item \textbf{Mejor paralelización}: Los bucles pueden ser optimizados por el compilador
\end{itemize}

El algoritmo mantiene la misma complejidad $O(N \log N)$ pero con constantes significativamente menores, lo que resulta en mejor rendimiento en la práctica.

\subsection{Transformada de Fourier Bidimensional}

Para la resolución numérica de la ecuación de onda en dos dimensiones, es necesario extender la Transformada de Fourier al caso
bidimensional. La Transformada Discreta de Fourier en 2D de una matriz $x[m,n]$ de dimensiones $M \times N$ se define como:

\begin{equation}
    \hat{x}[k,l] = \sum_{m=0}^{M-1} \sum_{n=0}^{N-1} x[m,n] e^{-i2\pi (km/M + ln/N)}
\end{equation}

donde $k = 0, 1, \ldots, M-1$ y $l = 0, 1, \ldots, N-1$.

La transformada inversa se expresa como:

\begin{equation}
    x[m,n] = \frac{1}{MN} \sum_{k=0}^{M-1} \sum_{l=0}^{N-1} \hat{x}[k,l] e^{i2\pi (km/M + ln/N)}
\end{equation}

Una propiedad fundamental de la FFT bidimensional es su separabilidad, que permite descomponer el cálculo en aplicaciones
consecutivas de FFT unidimensionales:

\begin{equation}
    \hat{x}[k,l] = \sum_{m=0}^{M-1} e^{-i2\pi km/M} \left[ \sum_{n=0}^{N-1} x[m,n] e^{-i2\pi ln/N} \right]
\end{equation}

Esto se puede implementar eficientemente mediante el siguiente algoritmo de dos pasos:

\begin{enumerate}
    \item \textbf{FFT por filas}: Aplicar FFT 1D a cada fila de la matriz de entrada:
          \begin{equation}
              \hat{y}[m,l] = \sum_{n=0}^{N-1} x[m,n] e^{-i2\pi ln/N}
          \end{equation}

    \item \textbf{FFT por columnas}: Aplicar FFT 1D a cada columna del resultado anterior:
          \begin{equation}
              \hat{x}[k,l] = \sum_{m=0}^{M-1} \hat{y}[m,l] e^{-i2\pi km/M}
          \end{equation}
\end{enumerate}

\subsubsection{Aplicación a la Ecuación de Onda}

En el contexto de la resolución de la ecuación de onda bidimensional, la FFT 2D permite transformar el operador Laplaciano $\nabla^2$
del dominio espacial al dominio frecuencial:

\begin{equation}
    \nabla^2 u(x,y) \xrightarrow{\text{FFT 2D}} -(\omega_x^2 + \omega_y^2) \hat{u}(\omega_x, \omega_y)
\end{equation}

donde $\omega_x = 2\pi k_x/L_x$ y $\omega_y = 2\pi k_y/L_y$ son las frecuencias espaciales discretas, y $L_x$, $L_y$ son las
dimensiones del dominio computacional.

Esta transformación convierte la ecuación diferencial parcial en una ecuación algebraica en el dominio frecuencial, facilitando
significativamente su resolución numérica mediante métodos espectrales.

\section{Metodología}
Se propone implementar un simulador físico que permita visualizar la evolución de una onda a traves de un campo. Para esto, se desarrollaron
las interfaces `WaveSimulation2D` y `WaveVisualizer` (ver seccion experimental). A su vez, la interfaz `WaveSimulation2D` se implemento en varios
backends distintos: Python, NumPy, C, C + ASM (Assembly), y C + AVX.

El objetivo es evaluar el rendimiento de cada implementación midiendo la variable \textit{steps per second} (pasos por segundo), que
indica cuántos pasos de simulación puede procesar cada backend en un segundo. Esta métrica es fundamental para evaluar la eficiencia
computacional de diferentes enfoques de implementación.

\subsection{Implementación Propuesta}
Con el objetivo de facilitar la experimentación, se propone utilizar un diseño comun a todos los backends. A modo de ejemplo, se muestra
la implementacion de uno de ellos:

\begin{verbatim}
    
    class ASMWaveSimulation2D:
        def __init__(self, size=256, domain_size=10.0, wave_speed=1.0, dt=0.01):
            self.c_core = c_backend_asm
            self._sim_ptr = self.c_core.create_simulation(size, domain_size, wave_speed, dt)
        
        def add_wave_source(self, x_pos, y_pos, amplitude=1.0, frequency=3.0, width=0.5):
            self.c_core.add_wave_source(self._sim_ptr, x_pos, y_pos, amplitude, frequency, width)
        
        def step(self):
            self.c_core.step_simulation(self._sim_ptr)
        
        def get_intensity(self):
            return self.c_core.get_intensity(self._sim_ptr)
        
        def get_real_part(self):
            return self.c_core.get_real_part(self._sim_ptr)
\end{verbatim}

La clase principal esta hecha en Python, porque facilita la visualización. Sin embargo, toda la logíca y el procesamiento se realiza
en C y Assembler. A continuacion se muestra un diagrama:
\begin{figure}[h]
    \centering
    \includegraphics[width=0.7\textwidth]{extra/diagram.png}
    \caption{Diagrama de arquitectura del simulador de ondas}
    \label{fig:wave_sim_architecture}
\end{figure} \\

\textbf{Initialize (C)} \\
Toma un tamaño de grilla, un tamaño del dominio, la velocidad de la hola y el intervalo de tiempo

\textbf{Add Wave Source (C)} \\
Toma la posicion de la ola a agregar a la simulacion.

\textbf{Step (C)} \\
Hace avanzar el tiempo de la simulacion.

\textbf{Get Intensity (C)} \\
Devuelve una grilla con la norma de la funcion en cada punto

\textbf{Get Real Part (C)} \\
Devuelve una grilla con la parte real de la funcion en cada punto, que sería la altura de la onda que veríamos en la vida real.

Como se ve en el diagrama, necesitamos calcular la transformada de fourier para cada paso de la simulacion. Es por esto que la
propuesta del trabajo es tratar de optimizar el algoritmo a distintos niveles y comparar sus rendimientos.

\subsection{Python y NumPy}

Para establecer un baseline de rendimiento, implementamos dos versiones en Python:

\textbf{Python puro:} La implementación utiliza el algoritmo iterativo FFT descrito en la sección 1.5, utilizando listas de números complejos. Esta versión sirve como referencia para entender cuánto más rápido funcionan las implementaciones en C y las librerías optimizadas como NumPy.

Esta implementación presenta limitaciones inherentes de Python: lentitud en operaciones numéricas y uso ineficiente de memoria, ya que las listas se almacenan de forma dispersa en memoria.

\textbf{NumPy:} Implementación optimizada que aprovecha las operaciones vectorizadas y bibliotecas optimizadas de álgebra lineal. En lugar de implementar la FFT manualmente, utilizamos directamente la función optimizada de NumPy:

\begin{verbatim}
def fft2(self, x):
    return np.fft.fft2(x)
\end{verbatim}

Como se observará en los resultados experimentales, la implementación de NumPy es extremadamente rápida y difícil de superar, aunque lograremos un rendimiento bastante similar con nuestras implementaciones optimizadas en C.

\subsection{C}
Como se ve en `ASMWaveSimulation2D`, la clase implementada en Python es simplemente una fachada, y toda la logica y estructuras de datos
utilizadas para correr la simulacion se manejan desde C. Esto funciona asi en el backend de C puro, como en el de C + AVX y C + Assembler.

En el archivo `c_backend.c` definimos dos structs fundamentales: Complex y WaveSimulation

\begin{verbatim}

typedef struct
{
    double real;
    double imag;
} Complex;
    
typedef struct
{
    Complex *wave;
    Complex *wave_k;
    double *grid_coords;
    double *k_grid_coords;
    double *K;
    int size;
    double domain_size;
    double wave_speed;
    double dt;
    double dx;
} WaveSimulation;

\end{verbatim}

WaveSimulation funciona de la siguiente manera

- \textbf{grid_coords} es una grilla cuadrada de un tamaño determinado (size). En cada posicion guarda un valor (x, y) que determina a que punto del dominio equivale
esa posicion. En la practica los valores se guardan todos contiguos en memoria [x0,y0,x1,y1,...].

- \textbf{k_grid_coords} es exactamente lo mismo pero en el dominio de las frecuencias. Recordemos que la transformada de Fourier permite resolver la ecuacion
diferencial facilmente al cambiar el dominio del espacio a las frecuencias.

- \textbf{wave} y \textbf{wave_k} funcionan en conjunto con grid_coords y k_grid_coords. Tienen, para cada valor de la grilla, el valor de la funcion en ese punto. Basicamente
wave[i][j] = phi(grid_coords[i][j]).

- \textbf{K}: COMPLETAR

- Los demas elementos son parametros que manejan el paso del tiempo y la velocidad de las olas. El primero no afecta la precision de la simulacion, dado que
la solucion utilizada es analitica. El segundo, wave_speed, es un parametro de la ecuacion diferencila.

Este struct funciona gracias a tres metodos principales: create_wave_simulation, add_wave_source y wave_sim_step.

\textbf{create_wave_simulation} \\
- Asigna las variable, realizando las reservas de memoria necesarias \\
- Inicializa grid_coords, k_grid_coords y K. \\

\textbf{add_wave_source} \\
- Agrega un circulito \\
- Aplica la transformada y vuelve a resolver la ecuacion \\

\textbf{wave_sim_step} \\
- Obtiene, para cada punto del dominio de las frecuencias, el nuevo valor \\
- Aplica la transformada inversa para obtener el nuevo valor de la funcion de onda \\

COMPLETAR: Cuando decimos (hace la transformada) en realidad llama a la transformada 2d.

Estos metodos son fundamentales y, como se menciono previamente, son la base necesaria que nos permite acelerar la simulacion en todos los backends. Ya aclarado
esto la implementacion de la transformada de Fourier:

\begin{verbatim}
static void fft_1d(Complex *x, int n, int inverse)
{
    assert(n > 0 && (n & (n - 1)) == 0 && "La longitud debe ser potencia de 2");

    bit_reverse(x, n);

    for (int len = 2; len <= n; len <<= 1)
    {
        double angle = 2.0 * M_PI / len * (inverse ? 1 : -1);
        Complex w = {cos(angle), sin(angle)};

        for (int i = 0; i < n; i += len)
        {
            Complex wn = {1.0, 0.0};
            for (int j = 0; j < len / 2; j++)
            {
                Complex u = x[i + j];
                Complex v = complex_mul(x[i + j + len / 2], wn);
                x[i + j] = complex_add(u, v);
                x[i + j + len / 2] = complex_sub(u, v);
                wn = complex_mul(wn, w);
            }
        }
    }

    if (inverse)
    {
        for (int i = 0; i < n; i++)
        {
            x[i].real /= n;
            x[i].imag /= n;
        }
    }
}
\end{verbatim}

\subsection{C + ASM}

La motivacion de esta optimizacion es que necesitamos calcular la transformada todo el tiempo para volver desde el dominio de las frecuencias a el dominio espacial,
y esto requiere calcular la transformada en cada paso de la simulacion.

La implementacion en Assembler puede entenderse facilmente haciendo un paralelismo linea por linea con la implementacion en C. Sin embargo, hay algunos detalles interesantes
que vale la pena mencionar.

\textbf{Uso de la pila x87 para operaciones matematicas}\\
La arquitectura x86-64 incluye una pila de registros especializada, conocida como la \textbf{pila x87}, diseñada para operaciones matemáticas en punto flotante.
Esta pila, compuesta por ocho registros de 80 bits (st0--st7), permite realizar cálculos complejos de manera eficiente, especialmente en operaciones trigonométricas
y exponenciales, que son fundamentales para la Transformada de Fourier.

En la implementación en Assembly, la pila x87 se utiliza para calcular funciones como seno y coseno de un ángulo, aprovechando instrucciones dedicadas como
\texttt{fsin} y \texttt{fcos}. El flujo típico consiste en cargar el ángulo en la pila, calcular el seno (dejando el ángulo aún disponible en la pila), almacenar
el resultado en memoria, y luego calcular el coseno sobre el mismo ángulo. Finalmente, ambos resultados se transfieren a registros xmm para su uso vectorizado.

El siguiente fragmento ilustra este proceso, equivalente a la operación en C \texttt{Complex w = {cos(angle), sin(angle)}}:
\begin{verbatim}    
.declarar_w:
fld     st0                             ; Copio el angulo devuelta en st0, st1 = angulo
fsin                                    ; st0 = sin(ang)   (ángulo sigue en st1)
fstp    qword [rsp]                     ; guardar sin en memoria
movhpd  xmm6, [rsp]                     ; xmm6 = [?, w_i]

fcos                                    ; st0 = cos(ang)
fstp    qword [rsp]                     ; guardar cos
movlpd   xmm6, [rsp]                     ; xmm6 = [w_r, w_i]        
; (pila x87 vacía)
\end{verbatim}
En este código, se observa cómo la pila x87 permite calcular ambas funciones trigonométricas sin necesidad de recalcular el ángulo ni acceder repetidamente a memoria, optimizando así el rendimiento en operaciones matemáticas intensivas.


\textbf{Representacion de numeros complejos} \\
Dado que la Transformada de Fourier trabaja con numeros complejos, es fundamental definir como vamos a manejarlos en Assembler. Como vimos en la implementacion de C, la
funcion recibe un puntero a un arreglo de complejos. Como ya vimos antes, el tipo de dato Complex ocupa 128 bits, o dos doubles, por lo que resulta ideal usar registros
xmm para operar con ellos. En este trabajo, trabajamos con numeros complejos en los registros xmm de la siguiente manera:

\begin{figure}[h]
    \centering
    \includegraphics[width=0.5\textwidth]{extra/xmm complex.png}
    \caption{Representación de números complejos en memoria para la implementación en Assembly}
    \label{fig:asm_complex_representation}
\end{figure}
Esta representacion no solo permite hacer solo una lectura de memoria por cada complejo, lo cual es razonable, si no que ademas permite aprovechar las
instrucciones de packed double para paralelizar sumas y restas:

\begin{verbatim}    
; --------------- x[i + j] = complex_add(u, v) --------------
movapd  xmm11, xmm0                     ; xmm11 = u_r, u_i
addpd   xmm11, xmm4                     ; xmm11 = u_r + v_r, u_i + v_i
movapd  [rdi],   xmm11
\end{verbatim}

Es decir, al sumar dos numeros complejos, la suma de la parte real e imaginaria se realiza simultaneamente.

\textbf{Definicion de la macro de multiplicacion} \\
Otro detalle interesante de la implementacion en Assembler es la utilizacion de una macro para el calculo de la multiplicacion compleja, dado de la misma se
utiliza mas de una vez. Esto resulta conveniente porque:\\

- Facilita la edicion del codigo si queremos hacer optimizaciones posteriores\\
- Al añadirse al codigo al momento de compilar, no tiene efectos en la performance, como si tendría utilizar una funcion.

La macro se define de la siguiente manera:

\begin{verbatim}
; Macro para multiplicación compleja: result = a * b
; Parámetros: a, b, result
; Fórmula: (a_r + a_i*i) * (b_r + b_i*i) = (a_r*b_r - a_i*b_i) + (a_r*b_i + a_i*b_r)*i
%macro COMPLEX_MUL 3
    movapd  %3, %1                      ; t1 = a
    mulpd   %3, %2                      ; t1 = [ar*br, ai*bi]
    xorpd   %3, [rel COMPLEX_NEGHI]     ; t1 = [ar*br, -(ai*bi)]

    movapd  xmm15, %1
    shufpd  xmm15, xmm15, 1   ; xmm15 = [ai, ar]
    mulpd   xmm15, %2         ; xmm15 = [ai*br, ar*bi]

    haddpd  %3, xmm15         ; %3 = [ar*br - ai*bi, ai*br + ar*bi]
%endmacro
\end{verbatim}

Y utiliza una mascara definida en la seccion .rodata que permite negar la parte imaginaria del numero.

\subsection{C + AVX}
Como se menciono en la seccion anterior, cada numero complejo ocupa 128 bits, por lo que el uso de los registros xmm, a pesar de permitir paralelizar las operaciones
efectivamente, solo soporta operar sobre un elemento del arreglo a la vez. Es por esto que se propone utilizar AVX para acelerar el ciclo interno de la transformada,
habitualmente llamado __butterfly__.

AVX2 (Advanced Vector Extensions 2) es una extensión del conjunto de instrucciones x86-64 introducida por Intel en 2013. Los registros introducidos duplican el ancho
de los registros SSE (128 bits), por lo que permiten operar sobre 4 valores double precision simultáneamente.

El uso de estos registros en este trabajo es analogo al explicado en la implementacion de C + ASM, solo que guardando dos numeros complejo por cada registro ymm:

\begin{figure}[h]
    \centering
    \includegraphics[width=0.7\textwidth]{extra/ymm complex.png}
    \caption{Esquema del ciclo butterfly vectorizado con AVX para la FFT}
    \label{fig:avx_butterfly}
\end{figure}

Para simplificar la implementacion y aprovechar las optimizaciones realizadas por el compilador, utilizamos C para añadir estas operaciones _inline_.

\subsubsection{Implementación del Ciclo Butterfly}

La implementación AVX mantiene la misma estructura algorítmica que las versiones anteriores, con la diferencia clave en el ciclo butterfly, donde se procesan
múltiples elementos simultáneamente. Las funciones que permiten esta aceleracion son

- \textbf{\_mm256\_loadu\_pd}\\
- \textbf{\_mm256\_setr\_pd} \\
- \textbf{complex\_mul\_simd}\\
- \textbf{\_mm256\_add\_pd} y \textbf{\_mm256\_sub\_pd}\\
- \textbf{\_mm256\_storeu\_pd}\\

Y vale la pena detenerse a considerar la implementacion de complex\_mul\_simd, dado que utiliza operaciones interesantes:
\begin{verbatim}
static inline __m256d complex_mul_simd(__m256d z1, __m256d z2)
{
    // z1 = [a1, b1, a2, b2], z2 = [c1, d1, c2, d2]
    // Result = [(a1*c1-b1*d1), (a1*d1+b1*c1), (a2*c2-b2*d2), (a2*d2+b2*c2)]
    
    __m256d ac_bd = _mm256_mul_pd(z1, z2);               // [a1*c1, b1*d1, a2*c2, b2*d2]
    __m256d z2_swapped = _mm256_shuffle_pd(z2, z2, 0x5); // [d1, c1, d2, c2]
    __m256d ad_bc = _mm256_mul_pd(z1, z2_swapped);       // [a1*d1, b1*c1, a2*d2, b2*c2]
    
    __m256d real_parts = _mm256_hsub_pd(ac_bd, ac_bd);   // [a1*c1-b1*d1, ...]
    __m256d imag_parts = _mm256_hadd_pd(ad_bc, ad_bc);   // [a1*d1+b1*c1, ...]
    
    return _mm256_unpacklo_pd(real_parts, imag_parts);
}
\end{verbatim}

\section{Experimentos}

Se realizaron experimentos para evaluar el rendimiento de cada backend implementado. Para cada backend, testeamos el correcto funcionamiento mediante una simulacion interactiva, y medimos
el rendimiento en distintos tamaños.

Los experimentos se realizaron en un sistema con las siguientes especificaciones:
\begin{itemize}
    \item \textbf{Procesador}: Intel x86-64 con soporte para AVX2
    \item \textbf{Sistema Operativo}: Linux 6.8.0-78-generic
    \item \textbf{Compilador}: GCC sin optimizaciones específicas (compilación por defecto)
    \item \textbf{Parámetros de simulación}:
          \begin{itemize}
              \item Tamaño del dominio: 8.0 unidades
              \item Velocidad de onda: 2.0 unidades/segundo
              \item Intervalo de tiempo: 0.02 segundos
              \item Pasos de simulación: 50 (para medición de rendimiento)
          \end{itemize}
\end{itemize}

Se evaluaron seis implementaciones diferentes:
\begin{enumerate}
    \item \textbf{Python}: Implementación en Python puro como baseline
    \item \textbf{NumPy}: Utilizando la biblioteca NumPy optimizada
    \item \textbf{C}: Implementación en C con optimizaciones del compilador
    \item \textbf{C + ASM}: C con rutinas críticas en Assembly x86-64
    \item \textbf{C + AVX}: Utilizando extensiones AVX para paralelización vectorial
\end{enumerate}

\subsection{Visualización Interactiva}
Obviamente, una parte fundamental del trabajo es poder visualizar interactivamente la simulacion. Por ese motivo, se implemento una visualización que permite colocar
agregar ondas clickeando en cualquier lugar del campo.

\begin{figure}[h]
    \centering
    \includegraphics[width=0.8\textwidth]{extra/live_visualization.png}
    \caption{Visualización interactiva del simulador de ondas 2D}
    \label{fig:live_visualization}
\end{figure}

\subsection{Rendimiento por Tamaño de Grilla}
Una vez verificado el correcto funcionamiento de cada backend, decidimos medir mas precisamente la performance. Para esto, dejamos de lado la visualizacion y simplemente
medimos la variable \textit{steps per second}. Basicamente, nos importa cuanto tarda en correr la funcion `step` en cada uno de los backends. Un parentesis importante es
que, a los efectos de la visualizacion, las impementaciones en C tienen un pequeño _overhead_ porque deben convertir su grilla a un numpy array y esto consume un tiempo extra.
En este trabajo evitamos lidiar con eso y simplemente medimos el tiempo que tarda en correr cada paso de la simulacion, porque es lo que decidimos optimizar.

A continuación se detallan los resultados:

\begin{table}[h]
    \centering
    \caption{Rendimiento de diferentes implementaciones (steps per second)}
    \label{tab:performance_results}
    \begin{tabular}{lccccccc}
        \toprule
        \textbf{Método} & \textbf{16×16}       & \textbf{32×32}       & \textbf{64×64}      & \textbf{128×128}    & \textbf{256×256}  & \textbf{512×512} & \textbf{1024×1024} \\
        \midrule
        Python          & 629,3                & 155,0                & 37,3                & 9,1                 & 2,1               & 0,5              & 0,1                \\
        ASM             & 58.661,6             & 17.405,2             & 4.397,7             & 1.088,5             & 243,1             & 58,0             & 13,4               \\
        C               & \underline{78.603,9} & 22.248,6             & 5.091,3             & 1.186,9             & 261,5             & 61,0             & 14,1               \\
        C\_AVX          & \textbf{82.695,3}    & \textbf{23.736,9}    & \underline{5.697,9} & \underline{1.335,2} & \underline{296,9} & \underline{68,7} & \underline{15,3}   \\
        Numpy           & 19.689,7             & \underline{12.918,3} & \textbf{5.707,2}    & \textbf{1.617,4}    & \textbf{380,6 }   & \textbf{85,4}    & \textbf{15,8}      \\
        \bottomrule
    \end{tabular}
\end{table}

\begin{figure}[h]
    \centering
    \begin{minipage}[t]{0.48\textwidth}
        \centering
        \includegraphics[width=\textwidth]{extra/steps_per_second.png}
        \caption{Comparación visual del rendimiento entre implementaciones de FFT y solver de ecuación de onda}
        \label{fig:performance}
    \end{minipage}
    \hfill
    \begin{minipage}[t]{0.48\textwidth}
        \centering
        \includegraphics[width=\textwidth]{extra/steps_per_second_top3.png}
        \caption{Comparación detallada del rendimiento en grillas grandes}
        \label{fig:performance_top3}
    \end{minipage}
\end{figure}

\subsection{Análisis de Resultados}

Los resultados experimentales revelan patrones interesantes en el rendimiento de las diferentes implementaciones. Como se observa en la Tabla \ref{tab:performance_results}, las Figuras \ref{fig:performance} y \ref{fig:performance_top3}, el rendimiento del backend en Python puro es significativamente inferior al resto, actuando como un baseline que demuestra la importancia de las optimizaciones.

\subsubsection{Rendimiento por Tamaño de Grilla}

Para grillas pequeñas (16×16 a 64×64), la implementación C+AVX muestra el mejor rendimiento, alcanzando hasta 82,080.3 steps/second en grillas de 16×16. Sin embargo, a medida que el tamaño de la grilla aumenta, NumPy comienza a superar a las implementaciones custom, especialmente en grillas grandes (512×512 y 1024×1024). La Figura \ref{fig:performance_top3} muestra con mayor detalle la competencia entre las tres mejores implementaciones en los tamaños más grandes, donde se puede apreciar claramente cómo NumPy toma la delantera en grillas de 512×512 y superiores.

\subsubsection{Comparación de Implementaciones}

\textbf{Python vs. Implementaciones Optimizadas:} La implementación en Python puro muestra un rendimiento dramáticamente inferior, con un factor de mejora de hasta 1,000x en grillas pequeñas comparado con las implementaciones optimizadas.

\textbf{C vs. ASM:} Contrariamente a la expectativa inicial, la implementación en Assembly puro (ASM) resulta ligeramente más lenta que la versión en C. Esto puede atribuirse a:
\begin{itemize}
    \item Optimizaciones avanzadas del compilador GCC con flags -O3
    \item Mejor manejo de registros y pipeline por parte del compilador
    \item Posibles ineficiencias en la implementación manual de Assembly
\end{itemize}

\textbf{C+AVX:} Esta implementación representa el mejor rendimiento entre las implementaciones custom, siendo en promedio 20-25\% más rápida que C puro. El uso de registros AVX de 256 bits permite procesar 4 elementos double precision simultáneamente.

\textbf{NumPy:} A pesar de ser una biblioteca de alto nivel, NumPy demuestra un rendimiento excepcional, especialmente en grillas grandes. Su implementación altamente optimizada, que probablemente utiliza BLAS/LAPACK y optimizaciones específicas de la arquitectura, la convierte en el líder en grillas de 512×512 y superiores.

\subsubsection{Escalabilidad}

Un aspecto notable es cómo las diferentes implementaciones escalan con el tamaño de la grilla. Mientras que las implementaciones custom mantienen un rendimiento relativamente constante en términos de steps/second, NumPy muestra una degradación más gradual, lo que sugiere una mejor optimización para problemas de gran escala.

\subsubsection{Conclusiones del Análisis}

\begin{enumerate}
    \item \textbf{Optimizaciones del compilador}: Las optimizaciones automáticas del compilador pueden superar implementaciones manuales en Assembly en muchos casos.
    \item \textbf{Paralelización vectorial}: Las extensiones AVX proporcionan mejoras significativas de rendimiento cuando se implementan correctamente.
    \item \textbf{Bibliotecas optimizadas}: NumPy demuestra que las bibliotecas altamente optimizadas pueden superar implementaciones custom, especialmente en problemas de gran escala.
    \item \textbf{Trade-off complejidad/rendimiento}: Las implementaciones más complejas (AVX) requieren más esfuerzo de desarrollo pero ofrecen mejor rendimiento.
\end{enumerate}

\section{Conclusiones}


\begin{thebibliography}{9}
    \bibitem{cooley1965algorithm}
    Cooley, J. W., \& Tukey, J. W. (1965). An algorithm for the machine calculation of complex Fourier series. \textit{Mathematics of computation}, 19(90), 297-301.

    \bibitem{frigo2005design}
    Frigo, M., \& Johnson, S. G. (2005). The design and implementation of FFTW3. \textit{Proceedings of the IEEE}, 93(2), 216-231.

    \bibitem{lawson1979basic}
    Lawson, C. L., et al. (1979). Basic linear algebra subprograms for Fortran usage. \textit{ACM Transactions on Mathematical Software}, 5(3), 308-323.

    \bibitem{dagum1998openmp}
    Dagum, L., \& Menon, R. (1998). OpenMP: an industry standard API for shared-memory programming. \textit{IEEE computational science and engineering}, 5(1), 46-55.

    \bibitem{muchnick1997advanced}
    Muchnick, S. (1997). \textit{Advanced compiler design and implementation}. Morgan Kaufmann.
\end{thebibliography}

\end{document}
